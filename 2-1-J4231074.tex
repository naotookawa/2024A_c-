\RequirePackage{plautopatch}
\RequirePackage[l2tabu, orthodox]{nag}
\documentclass[fontsize=11pt, jafontsize=11pt, paper=a4paper, dvipdfmx]{jlreq}
\usepackage{amsmath,amssymb}
\usepackage{newtxtext,newtxmath}
\usepackage{graphicx}
\usepackage{listings}
\lstset{%
 language={C++},
 % backgroundcolor={\color[gray]{.95}},%
 tabsize=2, % tab space width
 showstringspaces=false, % don't mark spaces in strings
 basicstyle={\ttfamily},%
 %identifierstyle={\small},%
 commentstyle={\itshape},%
 keywordstyle={\bfseries},%
 %ndkeywordstyle={\small},%
 stringstyle={\ttfamily},
 %frame={tb},
 breaklines=true,
 columns=[l]{fullflexible},%
 % numbers=left,%
 % numberstyle={\small},%
 xrightmargin=0zw,%
 %xleftmargin=3zw,%
 stepnumber=1,
 numbersep=1zw,%
 lineskip=-0.5ex%
}
%%%%%%%%%%%%%%%%%%%%%%%%%%%%%%%%%%%%%%%%%%%%%%%%%%%%%%%%%%%%%%%%%
% ここから上は(当面は)編集しないでください.
%%%%%%%%%%%%%%%%%%%%%%%%%%%%%%%%%%%%%%%%%%%%%%%%%%%%%%%%%%%%%%%%%

\title{レポート第  回 \\
  「」}
\author{氏 名:  大川 尚音 \\
        科類・クラス:理科1類37組 \\
        学生証番号:J4-231074 \\
        E-mail: na-okawa04@g.ecc.u-tokyo.ac.jp}
\date{\today}
%
\begin{document}
\maketitle
%
%%%%%%%%%%%%%%%%%%%%%%%%%%%%%%%%%%%%%%%%%%%%%%%%%%%%%%%%%%%%%%%%%
%%%%%%%%% 以下の文は,レポート提出時には削除してください.%%%%%%%%%%%%
%%%%%%%%%%%%%%%%%%%%%%%%%%%%%%%%%%%%%%%%%%%%%%%%%%%%%%%%%%%%%%%%%
今後,レポートは,本書式にしたがって作成するものとする.\underline{各課題ごとに},
プログラムリスト,実行結果,考察を書くこと.実際には,各課題ごとに
ファイル内の section 部(\%\% ここから \%\% から,\%\% ここまで \%\% まで)
をまるまるコピーペーストし,課題に合わせて編集するとよい.
%%%%%%%%%%%%%%%%%%%%%%%%%%%%%%%%%%%%%%%%%%%%%%%%%%%%%%%%%%%%%%%%%

\section{2.3節 例1}

\subsection{実行結果}
\label{sec:results1}

\begin{quote}
\begin{verbatim}
% ./2_3_1
-2.63255e+10
\end{verbatim}
\end{quote}
%
\subsection{考察}
ここに考察を入力

\section{2.3節 例2}

\subsection{実行結果}
\label{sec:results2}

\begin{quote}
\begin{verbatim}
% ./2_3_2
-2094914496
\end{verbatim}
\end{quote}
%
\subsection{考察}
ここに考察を入力

\section{2.4節 例1}

\subsection{実行結果}
\label{sec:results3}

\begin{quote}
\begin{verbatim}
% ./2_4_1
5.10064e+14
\end{verbatim}
\end{quote}
%
\subsection{考察}
ここに考察を入力

\section{2.4節 例2}

\subsection{実行結果}
\label{sec:results4}

\begin{quote}
\begin{verbatim}
% ./2_4_2
-2094914496
\end{verbatim}
\end{quote}
%
\subsection{考察}
ここに考察を入力

\section{2.4節 例3}

\subsection{実行結果}
\label{sec:results5}

\begin{quote}
\begin{verbatim}
% ./2_4_3
5.10064e+11
\end{verbatim}
\end{quote}
%
\subsection{考察}
ここに考察を入力

\section{2.7節 例1}

\subsection{実行結果}
\label{sec:results6}

\begin{quote}
\begin{verbatim}
% ./2_7_1
It is '3.141500'.
It is '   3.14'.
It is '3.142  '.
\end{verbatim}
\end{quote}
%
\subsection{考察}
ここに考察を入力

\section{2.8節 例1}

\subsection{実行結果}
\label{sec:results7}

\begin{quote}
\begin{verbatim}
% ./2_8_1
0.001
5.10064e+11

% ./2_8_1
1
5.10065e+14
\end{verbatim}
\end{quote}
%
\subsection{考察}
このように、cinを使うことによってterminal上でインタラクティブなコードを記述することができる。ここで、実際のコードは割愛するがdouble d = 0.001;の行をコメントアウトしたコードでは、コンパイル時にdに関してのエラーが発生したため、先に変数を定義することが必要であることが再確認できた。


\end{document}
