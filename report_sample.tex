\RequirePackage{plautopatch}
\RequirePackage[l2tabu, orthodox]{nag}
\documentclass[fontsize=11pt, jafontsize=11pt, paper=a4paper, dvipdfmx]{jlreq}
\usepackage{amsmath,amssymb}
\usepackage{newtxtext,newtxmath}
\usepackage{graphicx}
\usepackage{listings}
\lstset{%
 language={C++},
 % backgroundcolor={\color[gray]{.95}},%
 tabsize=2, % tab space width
 showstringspaces=false, % don't mark spaces in strings
 basicstyle={\ttfamily},%
 %identifierstyle={\small},%
 commentstyle={\itshape},%
 keywordstyle={\bfseries},%
 %ndkeywordstyle={\small},%
 stringstyle={\ttfamily},
 %frame={tb},
 breaklines=true,
 columns=[l]{fullflexible},%
 % numbers=left,%
 % numberstyle={\small},%
 xrightmargin=0zw,%
 %xleftmargin=3zw,%
 stepnumber=1,
 numbersep=1zw,%
 lineskip=-0.5ex%
}
%%%%%%%%%%%%%%%%%%%%%%%%%%%%%%%%%%%%%%%%%%%%%%%%%%%%%%%%%%%%%%%%%
% ここから上は(当面は)編集しないでください.
%%%%%%%%%%%%%%%%%%%%%%%%%%%%%%%%%%%%%%%%%%%%%%%%%%%%%%%%%%%%%%%%%

\title{レポート第 1 回 \\
  「レポートの書き方 \&{} C++プログラムの初歩」}
\author{氏 名: 金井 崇 \\
        科類・クラス:理科1類3組 \\
        学生証番号:J0-000000 \\
        E-mail: kanai@g.ecc.u-tokyo.ac.jp}
\date{\today}
%
\begin{document}
\maketitle
%
%%%%%%%%%%%%%%%%%%%%%%%%%%%%%%%%%%%%%%%%%%%%%%%%%%%%%%%%%%%%%%%%%
%%%%%%%%% 以下の文は,レポート提出時には削除してください.%%%%%%%%%%%%
%%%%%%%%%%%%%%%%%%%%%%%%%%%%%%%%%%%%%%%%%%%%%%%%%%%%%%%%%%%%%%%%%
今後,レポートは,本書式にしたがって作成するものとする.\underline{各課題ごとに},
プログラムリスト,実行結果,考察を書くこと.実際には,各課題ごとに
ファイル内の section 部(\%\% ここから \%\% から,\%\% ここまで \%\% まで)
をまるまるコピーペーストし,課題に合わせて編集するとよい.
%%%%%%%%%%%%%%%%%%%%%%%%%%%%%%%%%%%%%%%%%%%%%%%%%%%%%%%%%%%%%%%%%

%% ここから %%

\section{課題1 - 例1}
\subsection{プログラムリスト}
\label{sec:prog-list1}
% プログラムは 以下のように \begin{lstlisting} ... \end{lstlisting} を使って表示する.
% また,行番号をつけるときは
% \begin{lstlisting}[numbers=left]
% のようにオプションをつけよ.

\begin{lstlisting}
#include <iostream>
using namespace std;
int main() {
  cout << "Hello, Yama!" << endl;
}
\end{lstlisting}

%
\subsection{実行結果}
\label{sec:results1}

\begin{quote}           % 実行結果は \begin{quote} で字下げする
\begin{verbatim}
ca10101$ ./a.out
Hello, Yama!
ca10101$
\end{verbatim}
\end{quote}
%
\subsection{考察}
プログラムは,
\begin{verbatim}
\begin{lstlisting} ... \end{lstlisting}
\end{verbatim}
\ref{sec:prog-list1}節のように listings 環境を使って表示する.なお,
\begin{verbatim}
\lstinputlisting{file.cc}
\end{verbatim}
とすると,別に作成した file.cc を直接入力して表示させることも可能である.

また,行番号をつけるときは
\begin{verbatim}
\begin{lstlisting}[numbers=left,numberstyle=\ttfamily,xleftmargin=2zw]
\end{verbatim}
のようにオプションをつけよ(\ref{sec:prog-list2}節を参照のこと).

実行結果は,\ref{sec:results1}節のように quote 環境と verbatim 環境を使って表示する.

内容については,マウスの左ボタンを使ってカットアンドペーストするのが最も簡単だろう.

%% ここまで %%

\section{課題1 - 例2:和の計算}
\subsection{プログラムリスト}
\label{sec:prog-list2}

\begin{lstlisting}[numbers=left,numberstyle=\ttfamily,xleftmargin=2zw]
#include <iostream>
using namespace std;
int main() { /* wa no keisan */
  int  x, y, ans;
  cout << "x y -> ";
  cin >> x >> y;
  ans = x + y;
  cout << x << " + " << y << " = " << ans << endl;
}
\end{lstlisting}
%
\subsection{実行結果}
\begin{quote}           % 実行結果は \begin{quote} で字下げする
\begin{verbatim}
ca10101$ ./a.out
x y -> 10 20
10 + 20 = 30
ca10101$
\end{verbatim}
\end{quote}
%
\subsection{考察}
このプログラムは,和の計算をしている.
キーボードから \texttt{x, y} の2つの変数に値を入力すると,
それらの和が変数 \texttt{ans} に代入されて,
最終的に \texttt{x + y = ans} というように表示されている.
%
\end{document}
